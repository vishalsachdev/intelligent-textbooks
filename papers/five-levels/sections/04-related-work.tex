\section{Related Work}

The classification of educational technologies has been approached from multiple perspectives in the literature.

\subsection{Adaptive Learning Systems}

Research on adaptive learning systems dates to the 1970s with early intelligent tutoring systems (ITS). Woolf~\cite{woolf2010building} provides a comprehensive overview of ITS architectures, distinguishing between systems based on their student modeling capabilities. More recently, the rise of massive open online courses (MOOCs) has prompted new taxonomies focused on scalability and personalization~\cite{aleven2016instruction}.

\subsection{Educational Technology Maturity Models}

Several maturity models have been proposed for educational technology adoption. The Technology Integration Matrix (TIM)~\cite{harmes2016technology} classifies technology use along dimensions of student engagement and curricular integration. The SAMR model~\cite{puentedura2006samr} (Substitution, Augmentation, Modification, Redefinition) focuses on how technology transforms learning activities. However, these models address technology \textit{use} rather than technology \textit{capabilities}.

\subsection{AI in Education Frameworks}

The emergence of AI in education has prompted new classification efforts. Holmes et al.~\cite{holmes2019artificial} distinguish between AI applications that support learning management, learner support, and assessment. Luckin et al.~\cite{luckin2016intelligence} propose a framework centered on different types of intelligence that AI systems can exhibit in educational contexts. More recently, Haupt et al.~\cite{haupt2025deploying} examine the deployment of large language models and AI coding assistants in signal processing education, addressing challenges including hallucination mitigation, fairness, and the development of ``smart textbooks'' that integrate AI capabilities while maintaining transparency and trustworthiness.

\subsection{Privacy in Learning Analytics}

The learning analytics community has extensively studied privacy implications of educational data collection. Slade and Prinsloo~\cite{slade2013learning} outline ethical frameworks for learning analytics, while Rubel and Jones~\cite{rubel2016student} analyze student privacy through the lens of contextual integrity. However, these works do not provide systematic classification of systems by their data requirements.

\subsection{Gap in the Literature}

While existing frameworks address aspects of educational technology classification, none provide a comprehensive, capability-based taxonomy analogous to the SAE J3016 standard. Our framework fills this gap by defining clear levels based on system capabilities while explicitly linking each level to its data requirements and privacy implications.
