\section{Conclusion}

The rapid advancement of artificial intelligence is transforming educational content delivery. Without standardized frameworks for understanding these changes, educators, administrators, and policymakers struggle to make informed decisions about technology adoption.

This paper has proposed a five-level classification framework for intelligent textbooks, inspired by the SAE J3016 standard that successfully brought clarity to autonomous vehicle discourse. Our framework defines clear progression from static textbooks (Level 1) through interactive content (Level 2), adaptive systems (Level 3), chatbot integration (Level 4), to autonomous AI tutoring (Level 5).

A critical contribution of our analysis is the identification of Level 3 as a privacy inflection point. Below this threshold, intelligent textbooks can deliver significant educational value with minimal student data collection. At Level 3 and above, personalization requires increasingly detailed individual tracking, triggering regulatory obligations and institutional responsibilities that must be addressed proactively.

We have examined how existing educational technology standards---including xAPI, Learning Record Stores, and IEEE specifications---can provide interoperability foundations while supporting privacy through data portability and student control. However, gaps remain, particularly for conversational AI data and privacy-preserving computation.

The framework enables differentiated governance, with regulatory and institutional requirements scaled to the data intensity and AI autonomy of each level. This graduated approach avoids both over-regulation that stifles beneficial innovation and under-regulation that fails to protect student privacy.

As AI capabilities continue their exponential growth, the educational technology landscape will continue to evolve. The five-level framework provides a stable vocabulary for navigating this evolution---enabling clear communication, informed decision-making, and responsible innovation.

We invite the educational technology community to adopt, refine, and extend this framework. Just as SAE J3016 evolved through multiple revisions in response to technological and regulatory developments, we expect this framework to evolve as intelligent textbooks mature. The goal is not a permanent taxonomy but a useful starting point for the conversations that will shape the future of education.

It is the author's observation that many educational institutions currently suffer from a ``deer in the headlights'' syndrome. They acknowledge that AI is transforming education, yet remain paralyzed---unable to translate awareness into action. This inaction carries real risk: institutions that fail to adapt may find themselves overtaken by more action-oriented competitors who embrace intelligent textbooks and deliver superior learning experiences. It is the author's sincere hope that the five-level model provides small, incremental stepping stones that allow educational institutions to move forward with confidence. By starting with Level 2 pilot projects and progressing thoughtfully, institutions can build the expertise and infrastructure needed to thrive in an AI-enhanced educational landscape---ultimately giving their students the best education possible.

\begin{quote}
``All models are wrong, but some are useful.'' --- George Box
\end{quote}

We hope this model proves useful.
