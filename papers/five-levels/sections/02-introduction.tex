\section{Introduction}

The convergence of artificial intelligence and educational publishing is creating a new category of learning resources that defy traditional classification. Products marketed as ``intelligent,'' ``adaptive,'' or ``AI-powered'' textbooks vary enormously in their actual capabilities---from simple keyword search to sophisticated conversational tutoring. This ambiguity creates significant challenges for educational institutions attempting to evaluate, procure, and regulate these emerging technologies.

The problem is not merely semantic. Without clear classification, educators cannot make informed decisions about which tools best serve their pedagogical goals. Administrators struggle to develop appropriate policies for data governance. Policymakers lack the precision needed to craft effective regulations. And students remain uninformed about what data they are sharing and how it will be used.

This paper proposes a solution: a five-level classification framework for intelligent textbooks, modeled on the successful SAE J3016 standard for autonomous vehicles. Just as that framework transformed discourse around self-driving cars---enabling clear communication between manufacturers, regulators, and consumers---we argue that a similar framework can bring much-needed clarity to the educational technology landscape.

The genesis of this framework was a blog post written to initiate discussion within the educational technology community~\cite{mccreary2024fivelevels}. The response was substantial: dozens of educators, technologists, and policymakers provided valuable feedback that has shaped and refined the framework presented here. This paper represents the synthesis of those conversations, formalized for broader academic discourse.

Our framework is motivated by three converging trends:

\begin{enumerate}
    \item \textbf{Exponential AI Growth}: Large language models have progressed from experimental curiosities to production-ready tutoring systems in just a few years. Educational institutions need frameworks that can accommodate this pace of change.

    \item \textbf{Proliferating Products}: The market is flooded with AI-enhanced educational products, each claiming unique capabilities. Standardized classification enables meaningful comparison.

    \item \textbf{Privacy Imperatives}: Higher levels of textbook intelligence require increasingly detailed student data, raising questions that current regulatory frameworks were not designed to address.
\end{enumerate}

The remainder of this paper is organized as follows. Section 2 examines the economics of AI-generated educational content, presenting evidence from METR research that content creation costs are dropping exponentially. Section 3 reviews related work in educational technology classification. Section 4 examines the SAE J3016 autonomous vehicle standard and extracts lessons for educational technology. Section 5 presents our five-level framework in detail. Section 6 analyzes the privacy inflection point at Level 3. Section 7 discusses relevant educational technology standards. Section 8 addresses implementation considerations. Section 9 discusses limitations and future work. Section 10 provides recommendations for institutions beginning their intelligent textbook journey, including specific pilot project categories. Section 11 concludes.
